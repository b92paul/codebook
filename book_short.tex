\documentclass[10pt,twocolumn,oneside]{article}
\usepackage[T1]{fontenc}
\usepackage{amsmath, courier, listings, fancyhdr, graphicx}
\topmargin=0pt
\headsep=5pt
\textheight=750pt
\footskip=0pt
\voffset=-40pt
\textwidth=520pt
\marginparsep=0pt
\marginparwidth=0pt
\marginparpush=0pt
\oddsidemargin=0pt
\evensidemargin=0pt
\hoffset=-30pt
 
\lstset{
	language=C,
	basicstyle=\tt,
	breaklines=true,
	tabsize=2
}
 
\begin{document}
\pagestyle{fancy}
\fancyfoot{}
\fancyfoot[R]{\includegraphics[width=20pt]{taipei-hot.jpg}}
\fancyhead[L]{Taipei-Hot, National Taiwan University}
\fancyhead[R]{\thepage}
\renewcommand{\headrulewidth}{0.4pt}
\renewcommand{\contentsname}{Contents} 

\scriptsize
\tableofcontents

\section{source}
\subsection{default}
\begin{lstlisting}
#include <bits/stdc++.h>
#define openfile(s) freopen(s".in","r",stdin);freopen(s".out","w",stdout)
#define mpr std::make_pair
#define lg(x) (31-__builtin_clz(x))
#define __count __builtin_popcount
#define X first
#define Y second
#define mst(x) memset(x,0,sizeof(x))
#define mst1(x) memset(x,-1,sizeof(x))
#define ALL(c) (c).begin(),(c).end()
#define FOR(i,n) for(int i=0;i<n;i++)
#define FOR1(i,n) for(int i=1;i<=n;i++)
#define FORit(it,c) for(__typeof(c.begin()) it=c.begin();it!=c.end();++it)
#define pb push_back
#define RID(x) int x;RI(x)
using namespace std;
typedef long long LL;
typedef double LD;
typedef vector<int> VI;
typedef std::pair<int,int> PII;
template<class T>inline void maz(T &a,T b){if(a<b)a=b;}
template<class T>inline void miz(T &a,T b){if(a>b)a=b;}
template<class T>inline T abs(T a){return a>0?a:-a;}
void RI() {}
template<typename... T>
void RI( int& head, T&... tail ) {
    scanf("%d",&head);
    RI(tail...);
}
const int N=100100;

int main(){
	//openfile("");
	return 0;
}


\end{lstlisting}
\section{computational geometry}
\subsection{convexhull (與 chiang sheng-wen 衝突的複本 2015-01-18)}
\begin{lstlisting}
LL cross(unit x,unit a,unit b){
	return (a-x)^(b-x);
}
void convex_hull(){
	sort(t,t+n);
	top=0;
	for(int i=0;i<n;i++){
		while(top>=2&&cross(s[top-2],s[top-1],t[i])<=0)top--;
		s[top++]=t[i];
	}
	for(int i=n-2,k=top+1;i>=0;i--){
		while(top>=k&&cross(s[top-2],s[top-1],t[i])<=0)top--;
		s[top++]=t[i];
	}
	//top--;
}

\end{lstlisting}
\subsection{convexhull-3D}
\begin{lstlisting}
#include <bits/stdc++.h>
using namespace std;
typedef double LD;
const int N=110;
const LD EPS=1e-7;
int flag[N][N],n;
struct point{double x,y,z;
	point operator-(point &b)const
	{return (point){x-b.x,y-b.y,z-b.z};}
	LD len()const{return sqrt(x*x+y*y+z*z);}
};
struct face{int a,b,c;};
point t[N];
double dot(point a, point b)
{return a.x*b.x+a.y*b.y+a.z*b.z;}
point cross(point a, point b)
{return (point){a.y*b.z-b.y*a.z,a.z*b.x-b.z*a.x,a.x*b.y-b.x*a.y};}
point ver(face f)
{return cross(t[f.b]-t[f.a],t[f.c]-t[f.a]);}
double calc(point a,point b,point c,point d){
	point tmp=cross(d-b,c-b);
	return abs(tmp.x*(a.x-b.x)+tmp.y*(a.y-b.y)+tmp.z*(a.z-b.z));
}
double AREA(point a,point b,point c){
	point tmp=cross(b-a,c-a);
	return tmp.len()*0.5;
}
bool zero(point a){//a.len()<EPS?
	return abs(a.x)<EPS&&abs(a.y)<EPS&&abs(a.z)<EPS;
}
int main(){
	int i,j,w=1;
	while(~scanf("%d",&n)&&n){
		for(i=0;i<n;i++)
			scanf("%lf%lf%lf",&t[i].x,&t[i].y,&t[i].z);
		for(i=1;i<n;i++)
			if(!zero(t[i]-t[0])){swap(t[i],t[1]);break;}
		for(i++;i<n;i++)
			if(!zero(cross(t[1]-t[0],t[i]-t[0]))){swap(t[i],t[2]);break;}
		point tmp=cross(t[0]-t[1],t[2]-t[1]);
		for(i++;i<n;i++)
        	if(abs(dot(tmp,t[0]-t[i]))>EPS){swap(t[i],t[3]);break;}
		std::vector<face>now;
		now.push_back((face){0,1,2});
		now.push_back((face){2,1,0});
		for(i=3;i<n;i++){
			memset(flag,0,sizeof(flag));
			std::vector<face> next;
			for(j=0;j<now.size();j++){
				face& f=now[j];
				double d=dot(t[i]-t[f.a],ver(f));
				if (d<=0.0) next.push_back(f);
	            int ff=0;
				if(d>0.0)ff=1;
				else if(d<0.0)ff=-1;
	            flag[f.a][f.b]=flag[f.b][f.c]=flag[f.c][f.a]=ff;
	        }
			for(j=0;j<now.size();j++){
				face& f=now[j];
				if(flag[f.a][f.b]>0&&flag[f.a][f.b]!=flag[f.b][f.a])
					next.push_back((face){f.a,f.b,i});
				if(flag[f.b][f.c]>0&&flag[f.b][f.c]!=flag[f.c][f.b])
					next.push_back((face){f.b,f.c,i});
				if(flag[f.c][f.a]>0&&flag[f.c][f.a]!=flag[f.a][f.c])
					next.push_back((face){f.c,f.a,i});
	        }
	        now=next;
	    }
	    double volume=0.0;
	    for(i=0;i<now.size();i++){
			if(!now[i].a||!now[i].b||!now[i].c)continue;
			volume+=calc(t[0],t[now[i].a],t[now[i].b],t[now[i].c]);
		}
		//printf("%d\n",now.size());
		printf("%.2lf\n",volume/6.0);
		double area=0.0;
		for(i=0;i<now.size();i++)
			area+=AREA(t[now[i].a],t[now[i].b],t[now[i].c]);
		//printf("%.2lf\n",area);
	}
}

\end{lstlisting}
\subsection{convexhull}
\begin{lstlisting}
LL cross(PT x,PT a,PT b){
	return (a-x)^(b-x);
}
void convex_hull(PT t[], int n){
	sort(t,t+n);
	top=0;
	for(int i=0;i<n;i++){
		while(top>=2&&cross(s[top-2],s[top-1],t[i])<=0)top--;
		s[top++]=t[i];
	}
	for(int i=n-2,k=top+1;i>=0;i--){
		while(top>=k&&cross(s[top-2],s[top-1],t[i])<=0)top--;
		s[top++]=t[i];
	}
	//top--;
}

\end{lstlisting}
\subsection{functinos}
\begin{lstlisting}
const LD Magic=39281841021341.0;
const LD EPS = 1e-6;
bool same(LD a, LD b){return abs(a-b)<EPS;}
struct PT{
	LD x,y;
	PT(){}
	PT(LD X,LD Y):x(X),y(Y){}
	PT operator+(const PT &A)const{return PT(x+A.x,y+A.y);}
	PT operator-(const PT &A)const{return PT(x-A.x,y-A.y);}
	PT operator*(const LD &A)const{return PT(x*A,y*A);}
	LD operator^(const PT &A)const{return x*A.y-y*A.x;}
	PT norm()const{return PT(-y,x)*(1.0/len());}
	PT rot(LD a)const{
		return PT(cos(a)*x-sin(a)*y,
				  sin(a)*x+cos(a)*y);
	}
	bool operator<(const PT &A)const{//sort by theta
		int s1=(y?y:x)<0; // if(X+ || Y+ || face1,2)s=0
		int s2=(A.y?A.y:A.x)<0;
		return s1==s2?(x*A.y>y*A.x):(s1<s2);
	}
	LD len()const{return sqrt(x*x+y*y);}
	LD len2()const{return x*x+y*y;}
};
inline PT l2p(PT a,PT b,PT c,PT d) {
	PT v=d-c;
	if(abs(v^(b-a))<EPS)return PT(Magic,Magic);
	LD t=(v^(c-a))/(v^(b-a));
	return a+(b-a)*t;
}
LD area(){
	int a=0;//A[n]=A[0];
	for(int i=0;i<n;i++)
		a+=A[i]^A[i+1];
	return a/2.0;
}
PT centroid() {
	PT ret(0,0);
	FOR(i,n) {
		ret.x+=(A[i].x+A[i+1].x)*(A[i]^A[i+1]);
		ret.y+=(A[i].y+A[i+1].y)*(A[i]^A[i+1]);
	}
	return ret*(1.0/(6*area()));
}
LD simpson (LD a, LD b, int n) {
	//n must be even
	//x_j=a+jh
	LD h=(b-a)/n;
	\int_a^b f(x)dx = h/3*sigma(j=1~n/2){f(x_2j-2)+4f(x_2j-1)+f(x_2j)}
}

\end{lstlisting}
\subsection{HalfPlaneIntersection}
\begin{lstlisting}
const int N=501,INF=1<<26;
const LD EPS=1e-9;
const int mod=100000007;
int sgn(LD x){return (x>EPS)-(x<-EPS);}
struct PT {
    LD x,y;
    PT(LD X=0,LD Y=0):x(X),y(Y){}
    void read() {
        scanf("%lf%lf",&x,&y);
    }
    LD ang(){return atan2(y,x);}
    PT operator+(const PT &A)const {
        return PT(x+A.x,y+A.y);
    }
    PT operator-(const PT &A)const {
        return PT(x-A.x,y-A.y);
    }
    PT operator*(LD a)const{
        return PT(x*a,y*a);
    }
    PT operator/(const LD &A)const {
        return PT(x/A,y/A);
    }
    LD operator^(const PT &A)const {
        return x*A.y-y*A.x;
    }
    PT norm()const {
        return PT(-y,x);
    }
}p[N];
LD X(PT o, PT a,PT b){return (a-o)^(b-o);}
struct Ln {
    PT p1,p2;
    LD ang;
    Ln(){}
    Ln(PT P1,PT P2):p1(P1),p2(P2),ang((p2-p1).ang()){}
};
PT jiao(Ln a,Ln b) {
    LD u=X(a.p1,a.p2,b.p1);
    LD v=X(a.p2,a.p1,b.p2);
    return (b.p1*v+b.p2*u)/(u+v);
}
vector<Ln>L;
vector<PT>convex;
deque<Ln>Q;
bool cmp(Ln a, Ln b) {
    int ret = sgn(a.ang-b.ang);
    return ret!=0?ret>0:sgn(X(a.p1,a.p2,b.p1))<0;
}
bool check(Ln a, Ln b, Ln c) {
    //PT t=jiao(a,b);printf("jiao = %.2f, %.2f\n",t.x,t.y);
    return sgn(X(c.p1,c.p2,jiao(a,b)))>0;
}
void solve() {
    sort(ALL(L),cmp);
    FOR(i,L.size()) {
        if(i>0&&sgn(L[i].ang-L[i-1].ang)==0)continue;
        while(Q.size()>=2 && !check(Q[Q.size()-2],Q[Q.size()-1],L[i]))Q.pop_back();
        while(Q.size()>=2 && !check(Q[0],Q[1],L[i]))Q.pop_front();
        Q.pb(L[i]);
    }
    while(Q.size()>=2 && !check(Q[Q.size()-2],Q[Q.size()-1],Q[0]))Q.pop_back();
    while(Q.size()>=2 && !check(Q[0],Q[1],Q[Q.size()-1]))Q.pop_front();
    if(Q.size()<2)return;
    FOR(i,Q.size())convex.pb(jiao(Q[i],Q[(i+1)%Q.size()]));
}
int n;
#define LOG(a) (cerr<<__LINE__<<": "#a" = "<<(a)<<endl)
LD area() {
    LD ans=0;
    FOR(i,convex.size())ans+=convex[i]^convex[(i+1)%convex.size()];
    return fabs(ans)*0.5;
}
int main() {
    L.pb(Ln(PT(0,0),PT(10000,0)));
    L.pb(Ln(PT(10000,0),PT(10000,10000)));
    L.pb(Ln(PT(10000,10000),PT(0,10000)));
    L.pb(Ln(PT(0,10000),PT(0,0)));
    int n;scanf("%d",&n);
    FOR(i,n){
        PT a,b;a.read();b.read();
        L.pb(Ln(a,b));
    }
    solve();
    printf("%.1f\n",area());
}


\end{lstlisting}
\subsection{minimum circle}
\begin{lstlisting}
#include <bits/stdc++.h>
using namespace std;
struct M{double x,y;void read(){scanf("%lf%lf",&x,&y);}};
const double limit=1e-5;
M a[3],o,p;
double r;
double sdis(M a,M b)
{return (a.x-b.x)*(a.x-b.x)+(a.y-b.y)*(a.y-b.y);}
M O(int i,int j){
	M o;
	o.x=(a[i].x+a[j].x)/2.0,o.y=(a[i].y+a[j].y)/2.0;
	return o;
}
void make_c(){
	double x=sdis(a[0],a[1]),y=sdis(a[2],a[1]),z=sdis(a[0],a[2]);
	if(y+z<=x+limit)r=x/4.0,o=O(0,1);
	else if(x+z<=y+limit)r=y/4.0,o=O(2,1);
	else if(x+y<=z+limit)r=z/4.0,o=O(2,0);
	else{
		M m1=O(0,1),m2=O(1,2);
		double la1=a[0].x-a[1].x,lb1=a[0].y-a[1].y,lc1=-(la1*m1.x+lb1*m1.y);
		double la2=a[1].x-a[2].x,lb2=a[1].y-a[2].y,lc2=-(la2*m2.x+lb2*m2.y);
		o.x=(lb1*lc2-lb2*lc1)/(la1*lb2-la2*lb1);
		o.y=(la1*lc2-la2*lc1)/(lb1*la2-la1*lb2);
		r=sdis(o,a[0]);
	}
}
int main()
{
	int m;
	while(~scanf("%d",&m)){
		if(m==2){o.read();p.read();printf("%.3lf\n",sqrt(sdis(o,p)/4.0));continue;}
		for(int i=0;i<3;i++)a[i].read();
		make_c();m-=3;
		while(m--)
		{
			p.read();
			if(sdis(p,o)<=r+limit)continue;
			for(int i=0;i<3;i++){
				swap(p,a[i]);make_c();
				if(sdis(p,o)<=r+limit)break;
			}
		}
		printf("%.3lf\n",sqrt(r));
	}
}

\end{lstlisting}
\section{data structure}
\subsection{link-cut tree}
\begin{lstlisting}
//link cut tree with treap
//by david942j
#include <bits/stdc++.h>
const int N=100100;
struct node{
	int pos,sum,val,hval;
	node *L,*R,*par;
	node(){pos=val=sum=0;}
	node(int _pos,node *p,int v,int h,node *l,node *r){pos=_pos;hval=h;par=p;val=v;L=l;R=r;sum=v;}
	inline void update(){
		sum=L->sum+R->sum+val;
	}
	void print(){
		printf("node pos=%d, sum=%d\n",pos,sum);
	}
}*nil,*map[N];
int n;
int heap[N],par[N],val[N];
int real[N];
void init(){
	nil=new node();
	nil->par=nil->L=nil->R=nil;
	scanf("%d",&n);
	for(int i=1;i<=n;i++)
		heap[i] = i;
	std::random_shuffle(heap+1,heap+n+1);
	for(int i=1;i<=n;i++){
		scanf("%d%d",&par[i],&val[i]);
		map[i]=new node(i,nil,val[i],heap[i],nil,nil);
		real[i]=0;
	}
}
void changeval(int x,int v){
	node *now = map[x];
	for(now->val=v;now!=nil;now=now->par)
		now->update();
}
node *findroot(node *now){
	while(now->par!=nil)now=now->par;
	return now;
}
node* merge(node *a,node *b){
	if(a==nil)return b;
	if(b==nil)return a;
	if(a->hval > b->hval){
		a->R=merge(a->R,b);
		a->R->par=a;
		a->update();
		return a;
	}
	else{
		b->L=merge(a,b->L);
		b->L->par=b;
		b->update();
		return b;
	}
}
int LR[N];
void split(node* now,node *&a,node *&b){
	if(LR[now->pos]==-1){
		a=now->L;
		now->L=nil;
		now->update();
		b=now;
		return;
	}
	if(LR[now->pos]){
		split(now->R,now->R,b);
		now->update();
		now->R->par=now;
		a=now;
	}
	else{
		split(now->L,a,now->L);
		now->update();
		now->L->par=now;
		b=now;
	}
}
void cut(node *now,node *&a,node *&b){
	node *x=now->par;
	LR[now->pos]=-1;
	while(x!=nil){
		LR[x->pos] = (x->R==now);
		now=x;
		x=now->par;
	}
	split(now,a,b);
	a->par=b->par=nil;
}
void changepar(int x,int y){
	node *a=nil,*b=nil;
	if(real[par[x]]!=x)
		;//b=findroot(map[x]);
	else{
		cut(map[x],a,b);
		real[par[x]]=0;
	}
	par[x]=y;
}
node *head(node *now){
	while(now->L!=nil)now=now->L;
	return now;
}
void expose(node *x){
	x=findroot(x);
	node *h=head(x);
	while(par[h->pos]!=0){
		int &r=real[par[h->pos]];
		if(r==0){
			r=h->pos;
			x=merge(findroot(map[par[h->pos]]),x);
			h=head(x);
		}
		else{
			node *a=nil,*b=nil;
			cut(map[r],a,b);
			r=h->pos;
			x=merge(a,x);
			h=head(x);
		}
	}
}
void querylength(int u,int v){
	//printf("query %d %d\n",u,v);
	node *x=map[u],*y=map[v],*a=nil,*b=nil,*c=nil;
	expose(x);
	expose(y);
	if(findroot(x)==findroot(y)){
		cut(y,a,b);
		int sum1=b->sum;
		merge(a,b);
		cut(x,a,b);
		int sum2=b->sum;
		merge(a,b);
		if(sum1>sum2)
			printf("%d\n",sum1-sum2+x->val);
		else printf("%d\n",sum2-sum1+y->val);
	}
	else{
		node *lca=map[par[head(findroot(x))->pos]];
		//lca->print();
		cut(lca,a,b);
		cut(y,b,c);
		int sum1=b->sum+y->val;
		b=merge(a,b);
		merge(b,c);
		cut(x,a,b);
		printf("%d\n",a->sum+x->val+sum1);
		merge(a,b);
	}
}
void solve(){
	int m,t,x,y;
	scanf("%d",&m);
	while(m--){
		scanf("%d%d%d",&t,&x,&y);
		if(t==1)
			querylength(x,y);
		else if(t==2)
			changeval(x,y);
		else
			changepar(x,y);
	}
}
int main(){
	init();
	solve();
}

\end{lstlisting}
\subsection{magic (與 chiang sheng-wen 衝突的複本 2015-01-18)}
\begin{lstlisting}
#include <bits/stdc++.h>
#include <ext/rope>
#include<ext/pb_ds/assoc_container.hpp>
using namespace std;
using namespace __gnu_cxx;
using namespace __gnu_pbds;
typedef std::pair<int,int> PII;
typedef tree<int,int,less<int>,rb_tree_tag,tree_order_statistics_node_update> TR;
const int N=100000;
rope<int>a;
int t[N],n;
void _rope(){
	a=rope<int>(t,t+n);
	//operator same as string
}
void _ordered_map(){
	static TR t;
    t.clear();
    int m;scanf("%d",&m);
    while(m--){
		char s[3],x;
		scanf("%s%d",s,&x);
		// query the xth(0-base) key
		if(s[0]=='?')printf("%d\n",t.find_by_order(x)->first);
		else if(s[0]=='I') {
			int y;scanf("%d",&y);
			t[x]=y;
		}
		else if(s[0]=='D') t.erase(x);
	}
}
int main(){
	_ordered_map();
    return 0;
}

\end{lstlisting}
\subsection{magic}
\begin{lstlisting}
#include <bits/stdc++.h>
#include <ext/rope>
#include<ext/pb_ds/assoc_container.hpp>
using namespace std;
using namespace __gnu_cxx;
using namespace __gnu_pbds;
typedef std::pair<int,int> PII;
typedef tree<int,int,less<int>,rb_tree_tag,tree_order_statistics_node_update> TR;
const int N=100000;
rope<int>a;
int t[N],n;
void _rope(){
	a=rope<int>(t,t+n);
	//operator same as string
}
void _ordered_map(){
	static TR t;
    t.clear();
    int m;scanf("%d",&m);
    while(m--){
		char s[3],x;
		scanf("%s%d",s,&x);
		// query the xth(0-base) key
		if(s[0]=='?')printf("%d\n",t.find_by_order(x)->first);
		else if(s[0]=='I') {
			int y;scanf("%d",&y);
			t[x]=y;
		}
		else if(s[0]=='D') t.erase(x);
	}//t.order_of_key(int) // 0-base, lower_bound;
}
int main(){
	_ordered_map();
    return 0;
}
//{{TwT
#include <bits/extc++.h>
using namespace std;
using namespace __gnu_pbds;
typedef tree<int,null_type,less<int>,rb_tree_tag,
tree_order_statistics_node_update> set_t; typedef __gnu_pbds::priority_queue<int> heap_t; int main() {
    set_t s;
    for ( int i=0; i<5; i++ ) s.insert(10*i); assert(*s.find_by_order(0) == 0); assert(*s.find_by_order(3) == 30); assert(s.find_by_order(5) == s.end()); assert(s.order_of_key(0) == 0); assert(s.order_of_key(30) == 3); assert(s.order_of_key(35) == 4); assert(s.order_of_key(100) == 5);
    heap_t h1,h2;
    for ( int i=1; i<=10; i+=2 ) h1.push(i); for ( int i=2; i<=10; i+=2 ) h2.push(i); assert(h1.size() == 5);
    assert(h1.top() == 9);
    assert(h2.size() == 5);
    assert(h2.top() == 10);
    h1.join(h2);
    assert(h1.size() == 10);
    assert(h1.top() == 10);
    return 0;
}

\end{lstlisting}
\subsection{special treap}
\begin{lstlisting}
void split(node *now,int x,node* &a,node* &b){
	//split first x nodes
	if(now==nil){a=b=nil;return;}
	now->sign_down();
	if(now->L->size+1 <= x){
		a=now;
		split(now->R,x-now->L->size-1,a->R,b);
		a->update();
	}
	else{
		b=now;
		split(now->L,x,a,b->L); 
		b->update();
	}
}
node* merge(node *a,node *b){
	if(a==nil)return b;
	else if(b==nil)return a;
	if( b->hval <= a->hval){
		a->sign_down();
		a->R = merge(a->R,b);
		a->update();
		return a;
	}
	else{
		b->sign_down();
		b->L = merge(a,b->L);
		b->update();
		return b;
	}
}

\end{lstlisting}
\subsection{splay tree}
\begin{lstlisting}
#include <cstdio>
#include <cstdlib>
#include <iostream>
using namespace std;
const int N=500010;
inline int max(int a,int b){return a>b?a:b;}
struct node{
	int val,sum,maxL,maxR,maxS,size;
	bool rev,same;
	node *ch[2],*pre;
}tree[N],*root,*Nil;
node* newnode(int val){
	node *tmp=new node;
	tmp->val=tmp->sum=tmp->maxL=tmp->maxR=tmp->maxS=val;
	tmp->rev=tmp->same=false;
	tmp->pre=tmp->ch[0]=tmp->ch[1]=Nil;
	return tmp;
}
void make_same(node *now,int c){
	if(now==Nil)return;
	now->same=true;now->val=c;
	now->sum=c*now->size;
    now->maxL=now->maxR=now->maxS=max(now->sum,c);
}
void rev(node *now){
	if(now==Nil)return;
	swap(now->ch[0],now->ch[1]);
	swap(now->maxL,now->maxR);
	now->rev^=1;
}
void sign_down(node *now){
	if(now==Nil)return;
	if(now->same)
	{now->same=false;make_same(now->ch[0],now->val);make_same(now->ch[1],now->val);}
	if(now->rev)
	{now->rev=false;rev(now->ch[0]);rev(now->ch[1]);}
}
void update(node *now){
	node *L=now->ch[0],*R=now->ch[1];
	now->size=L->size+R->size+1;
	now->sum=L->sum+R->sum+now->val;
	now->maxL=max(L->maxL,L->sum+max(0,R->maxL)+now->val);
	now->maxR=max(R->maxR,R->sum+max(0,L->maxR)+now->val);
	now->maxS=max(max(L->maxS,R->maxS),max(0,L->maxR)+now->val+max(0,R->maxL));
}
void ror(node *now,int p) {// p=0 left ror
	node *par=now->pre;
	sign_down(par);sign_down(now);
	par->ch[!p]=now->ch[p];
	if(now->ch[p]!=Nil)now->ch[p]->pre=par;
	now->ch[p]=par;
	now->pre=par->pre;
	if(now->pre!=Nil)now->pre->ch[now->pre->ch[1]==par]=now;
	par->pre=now;
	if(par==root)root=now;
	update(par);
}
void splay(node *now,node *fa){
	for(sign_down(now);now->pre!=fa;)
		if(now->pre->pre)
			ror(now,now->pre->ch[0]==now);
		else{
			node *par=now->pre,*gpar=par->pre;
			if(gpar->ch[0]==par)
				if(par->ch[0]==now)
					{ror(par,1);ror(now,1);}
				else
					{ror(now,0);ror(now,1);}
			else
				if(par->ch[0]==now)
					{ror(par,0);ror(now,0);}
				else
					{ror(now,1);ror(now,1);}
		}
	update(now);
}
void find(int k,node *fa){
	node *tmp=root;
	while(1){
		sign_down(tmp);
		int s=tmp->ch[0]->size;
		if(s+1==k)break;
		if(k<=s)tmp=tmp->ch[0];
		else {
			tmp=tmp->ch[1];
			k-=s+1;
		}
	}
	splay(tmp,fa);
}
int t[N];
node* make_tree(int L,int R,node *par){
	if(L>R)return Nil;
	int mid=L+R>>1;
	node *tmp=newnode(t[mid]);
	tmp->ch[0]=make_tree(L,mid-1,tmp);
	tmp->ch[1]=make_tree(mid+1,R,tmp);
	tmp->pre=par;update(tmp);
	return tmp;
}
void Insert(int pos,int n){
	for(int i=1;i<=n;i++)scanf("%d",&t[i]);
	find(pos,Nil);find(pos+1,root);
	root->ch[1]->ch[0]=make_tree(1,n,root->ch[1]);
	splay(root->ch[1]->ch[0], Nil);
}
void del(node *now){
	if(now==Nil)return;
	del(now->ch[0]);
	del(now->ch[1]);
	delete now;
}
int main(){
	int n,m,pos,tot,c;
	char s[30];
	Nil = newnode(-(1<<20));Nil->size=Nil->sum=0;
	root = newnode(-(1<<20));
	root->ch[1]=newnode(-(1<<20));
	root->ch[1]->pre=root;
	update(root);
	scanf("%d%d",&n,&m);
	Insert(1,n);
	while(m--){
		scanf("%s",s);
		if(s[2]!='X')scanf("%d%d",&pos,&tot);
		if(s[2]=='K'){//[pos,pos+tot-1] = c
			scanf("%d",&c);
			find(pos,Nil);find(pos+tot+1,root);
			make_same(root->ch[1]->ch[0],c);splay(root->ch[1],Nil);
		}
		if(s[0]=='I')Insert(pos+1,tot);
		else if(s[0]=='D'){//delete
			find(pos,Nil);find(pos+tot+1,root);
			del(root->ch[1]->ch[0]);
			root->ch[1]->ch[0]=Nil;
			splay(root->ch[1],Nil);
		}
		else if(s[0]=='R'){//reverse
			find(pos,Nil);find(pos+tot+1,root);
			rev(root->ch[1]->ch[0]);splay(root->ch[1],Nil);
		}
		else if(s[0]=='G')//get-sum
		{
			find(pos,Nil);find(pos+tot+1,root);
			printf("%d\n",root->ch[1]->ch[0]->sum);
		}
		else if(s[2]=='X')//get-all-max_subsum
			printf("%d\n",root->maxS);
	}
}


\end{lstlisting}
\section{DP}
\subsection{connection3 DP}
\begin{lstlisting}
#include <bits/stdc++.h>
const int N=13,hash=999991;
const long long mod=1125899906842624LL;
typedef struct node{long long val;int p;struct node *next;}node;
typedef struct unit{char c[13];int p;struct unit *next;}unit;
unit h[hash];
node H[2][hash];
std::queue<int>Q[2];
int m;
char map[N][N];
bool ok(int n){
	int S[20],stop=0;
	for(;n;n>>=2){
		if(!(n&3))continue;
		if((n&3)==1)S[stop++]=1;
		else if(stop==0)return false;
		else stop--;
	}
	if(stop)return false;
	return true;
}
void DFS(int now,int k){
	if(k==0){
		if(ok(now)){
			unit *tmp=new unit;
			tmp->p=now;tmp->next=h[now%hash].next;
			h[now%hash].next=tmp;
			int S[20],stop=0,j=0;
			for(;now;now>>=2,j++){
				if(!(now&3))continue;
				if((now&3)==1)S[stop++]=j;
				else{
					tmp->c[j]=S[stop-1];
					tmp->c[S[stop-1]]=j;
					stop--;
				}
			}
		}
		return;
	}
	DFS(now<<2,k-1);
	DFS((now<<2)+1,k-1);
	DFS((now<<2)+2,k-1);
}
inline void add(int x,int y,long long c,node H[]){
	node *tmp=new node;
	tmp->p=y;tmp->val=c;
	tmp->next=H[x].next;
	H[x].next=tmp;
}
inline int C(int p,int j){return p<<(2*j);}
inline int z(int x,int j){return (x>>(2*j))&3;}
inline void flagnize(int y,long long val,node H[],bool f){
	for(node *tmp=H[y%hash].next;tmp;tmp=tmp->next)
		if(tmp->p==y){
			tmp->val+=val;
			if(tmp->val>=mod)tmp->val-=mod;
			return;
		}
	add(y%hash,y,val,H);
	Q[f].push(y);
}
long long get(int x,node H[]){
	for(node *tmp=H[x%hash].next;tmp;tmp=tmp->next)
		if(tmp->p==x)return tmp->val;
	return 0;
}
int cor(int x,int j){
	for(unit *tmp=h[x%hash].next;tmp!=NULL;tmp=tmp->next)
		if(tmp->p==x)
			return tmp->c[j];
}
void del(int x,node H[]){
	for(node *tmp=&H[x%hash];tmp->next!=NULL;tmp=tmp->next)
		if(tmp->next->p==x){
			node *tmp2=tmp->next;
			tmp->next=tmp2->next;
			delete tmp2;
			return;
		}
}
int main(){
	DFS(0,13);
	for(int i=0;i<hash;i++)H[0][i].next=H[1][i].next=NULL;
	int n,lastx,lasty;
	while(~scanf("%d%d",&n,&m)){
		for(int i=0;i<n;i++)
			scanf("%s",map[i]);
		lastx=-1;lasty=-1;
		for(int i=0;i<n;i++)
			for(int j=0;j<m;j++)
				if(map[i][j]=='.')
					lastx=i,lasty=j;
		bool f=0;
		Q[0].push(0);
		add(0,0,1,H[0]);
		for(int i=0;i<n;i++){
			while(!Q[f].empty()){
				int x=Q[f].front();Q[f].pop();
				if(z(x,m)==0)
					flagnize(x<<2,get(x,H[f]),H[!f],!f);
				del(x,H[f]);
			}
			f^=1;
			for(int j=0;j<m;j++){
				if(map[i][j]=='*'){
					while(!Q[f].empty()){
						int x=Q[f].front();Q[f].pop();
						if(!z(x,j)&&!z(x,j+1))
							flagnize(x,get(x,H[f]),H[!f],!f);
						del(x,H[f]);
					}
				}
				else{
					if(i==lastx&&j==lasty){
						while(!Q[f].empty()){
							int x=Q[f].front();Q[f].pop();
							if(z(x,j)==1&&z(x,j+1)==2)
								flagnize(x^C(z(x,j),j)^C(z(x,j+1),j+1),get(x,H[f]),H[!f],!f);
							del(x,H[f]);
						}
						f^=1;
						break;
					}
					while(!Q[f].empty()){
						int x=Q[f].front();Q[f].pop();
						long long v=get(x,H[f]);
						if(!z(x,j)&&!z(x,j+1))
							flagnize(x^C(1,j)^C(2,j+1),v,H[!f],!f);
						else if(!z(x,j)||!z(x,j+1)){
							flagnize(x,v,H[!f],!f);
							flagnize(x^( C( z(x,j) , j ) ^ C( z(x,j) , j+1 ) )^( C( z(x,j+1) , j ) ^ C( z(x,j+1) , j+1 ) ),v,H[!f],!f);
						}
						else if(z(x,j)&z(x,j+1))
							flagnize(x^C(z(x,j),j)^C(z(x,j+1),j+1)^C(3,cor(x,j+2-z(x,j))),v,H[!f],!f);
						else if(z(x,j)==2)
							flagnize(x^C(z(x,j),j)^C(z(x,j+1),j+1),v,H[!f],!f);
						del(x,H[f]);
					}
				}
				f^=1;
			}
			if(i==lastx)break;
		}
		printf("%lld\n",get(0,H[f]));
		while(!Q[0].empty()){int x=Q[0].front();Q[0].pop();del(x,H[0]);}
		while(!Q[1].empty()){int x=Q[1].front();Q[1].pop();del(x,H[1]);}
	}
}

\end{lstlisting}
\subsection{slope}
\begin{lstlisting}
#include <cstdio>
typedef long long LL;
int N, K,deq[1005], hd, ed;
LL pos[1005], val[1005]; 
LL dp[1005][1005];
LL sumVal[1005],sumPV[1005];
LL gSumVal(int i, int j) {
	if (j < i) return 0;
	return sumVal[j + 1] - sumVal[i];
}
LL gSumPV(int i, int j) {
	if (j < i) return 0;
	return sumPV[j + 1] - sumPV[i]; 
}
LL calc(int j, int i, int k) {
	return dp[j][k] + pos[i] * gSumVal(j + 1, i - 1) - gSumPV(j + 1, i - 1);
}
double slope(int i, int j, int k) {
	return ((double)(dp[i][k] - dp[j][k]) + (double)gSumPV(j + 1, i)) / (double)(gSumVal(j + 1, i));
}
int main() {
	while (~scanf("%d%d", &N, &K)) {
		for (int i = 0; i < N; ++i) {
			scanf("%lld%lld", &pos[i], &val[i]);
			if (i) {
				sumVal[i + 1] = sumVal[i] + val[i];
				sumPV[i + 1] = sumPV[i] + val[i] * pos[i];
			} else {
				sumVal[i + 1] = val[i];
				sumPV[i + 1] = val[i] * pos[i]; 
			}
		}
		dp[0][1] = 0;
		for (int i = 1; i < N; ++i) 
			dp[i][1] = dp[i - 1][1] + gSumVal(0, i - 1) * (pos[i] - pos[i - 1]); 
		for (int k = 2; k <= K; ++k) {
			dp[k - 1][k] = 0;
			hd = ed = 0;
			deq[ed++] = k - 2;
			deq[ed++] = k - 1;
			for (int i = k; i < N; ++i) {
				while (hd + 1 < ed && calc(deq[hd], i, k - 1) > calc(deq[hd + 1], i, k - 1))
					++hd;
				dp[i][k] = calc(deq[hd], i, k - 1);
                while (hd + 1 < ed && slope(i, deq[ed - 2], k - 1) > slope(i, deq[ed - 1], k - 1))
                    --ed;
				deq[ed++] = i;
			}
		}
		printf("%lld\n", dp[N - 1][K]);
	}
}

\end{lstlisting}
\section{grpah theory}
\subsection{BCC}
\begin{lstlisting}
//by david942j
#include <cstdio>
#include <cstdlib>
#include <vector>
#include <stack>
#include <algorithm>
#define mpr std::make_pair
#define X first
#define Y second
#define ALL(c) (c).begin(),(c).end()
#define FOR(i,n) for(int i=0;i<n;i++)
#define FOR1(i,n) for(int i=1;i<=n;i++)
#define pb push_back
using namespace std;
typedef long long LL;
typedef vector<int> VI;
typedef std::pair<int,int> PII;
template<class T>inline void maz(T &a,T b){if(a<b)a=b;}
template<class T>inline void miz(T &a,T b){if(a>b)a=b;}
const int N=100000;
vector<PII> G[N];
int n,m;
vector<PII> edge;
VI bcc[N],cut;
int btop,d;
bool flag[N];
int dfn[N],low[N];
stack<int>S;
void add(int x,int y) {
    int etop=edge.size();
    G[x].pb(mpr(y,etop));
    G[y].pb(mpr(x,etop));
    edge.pb(mpr(x,y));
}
void init() {
    edge.clear();
    cut.clear();
    while(!S.empty())S.pop();
    d=0;
    FOR(i,btop)bcc[i].clear();
    btop=0;
    FOR1(i,n){
        flag[i]=false;
        G[i].clear();
    }
}
int root=1;
void DFS(int x,int par) {
    flag[x]=true;
    dfn[x]=low[x]=++d;
    int son=0;
    FOR(i,G[x].size()) {
        int y=G[x][i].X;
        if(y==par)continue;
        if(!flag[y]) {
            S.push(G[x][i].Y);//push edge id
            DFS(y,x);
            son++;
            if(low[y]>=dfn[x]) { //cut
                if(x!=root)cut.pb(x);//root special
                while(true) {
                    int id=S.top();S.pop();
                    bcc[btop].pb(id);
                    if(id==G[x][i].Y)break;
                }
                btop++;
            }
            miz(low[x],low[y]);
        }
        else {
            if(dfn[y] < dfn[x])S.push(G[x][i].Y);//back edge
            miz(low[x],dfn[y]);
        }
    }
    if(x==root && son>1) cut.pb(x);//root is cut
}
void output() {
    puts("cut");
    sort(ALL(cut));cut.resize(unique(ALL(cut))-cut.begin());
    FOR(i,cut.size())printf("%d ",cut[i]);
    puts("");
    printf("%d bcc\n",btop);
    FOR(i,btop) {
        FOR(j,bcc[i].size())
        printf("(%d,%d)",edge[bcc[i][j]].X,edge[bcc[i][j]].Y);
        puts("\n--");
    }
}
void BCC() {//multiedge seem as one edge
    DFS(root,0);
}
int main() {
    while(~scanf("%d%d",&n,&m)) {
        init();
        while(m--) {
            int x,y;scanf("%d%d",&x,&y);//1~n
            add(x,y);
        }
        BCC();
        output();
    }
}
/*
 3 3
 1 2 2 3 3 1
 
 7 9
 1 2 1 4 1 5
 2 3 2 4
 3 4
 5 6 5 7
 6 7
 
 7 10
 1 2 1 4 1 5
 2 3 2 4
 3 4
 5 6 5 7
 6 7
 2 6
 
 4 3
 2 1 2 3 2 4
 
 */

\end{lstlisting}
\subsection{Bron-Kerbosch2}
\begin{lstlisting}
//by david942j  
//Bron_Kerbosch algorithm  
#include <cstdio>
#include <algorithm>
#include <cstdlib>
#define lg(x) 31-__builtin_clz(x)
using namespace std;
const int N=31;
int conn[N];
int n,ans;
void Search(int Pv,int Xv){
    if( Pv==0 ){
        if(Xv==0)
            ans++;
        return;
    }
    int B=Pv&~conn[lg(Pv&-Pv)];
    while(B){
        int i=lg(B&-B);
        Search(Pv&conn[i],Xv&conn[i]);
        Pv^=1<<i;
        Xv|=1<<i;
        B^=1<<i;
    }
}
void Bron_Kerbosch(){
    int Pv=(1<<n)-1,Xv=0;
    for(int i=0;i<n;i++){
        Search(Pv&conn[i],Xv&conn[i]);
        Pv^=1<<i;
        Xv|=1<<i;
    }
}
int main(){
    int m,T;
    scanf("%d",&T);
    while(T--){
        scanf("%d%d",&n,&m);
        /*for(int i=0;i<n;i++)
            conn[i]=((1<<n)-1)^(1<<i);*/
        for(int i=0;i<n;i++)conn[i]=0;
		while(m--){
            int x,y;
            scanf("%d%d",&x,&y);
            x--;y--;
            //if(conn[x]&(1<<y)){
                conn[x]|=1<<y;
                conn[y]|=1<<x;
            //}
        }
        ans=0;
        Bron_Kerbosch();
        printf("%d\n",ans);
    }
}
/*
define ^^ intersection
function Search(R,P,X)
	if P.empty() && X.empty()
		record R
		return
	choose a vertex u belongs to P U X
	for v in P\N(u) do
		Search(R U {v}, P^^N(v),X^^N(v))
		P=P\{v}
		X=X U {v}
	end
end

function Bron_Kerbosch(){
	R=0;P=V(G);X=0
    for v in degeneracy order of G {
        Search(R U {v},P^^N(v),X^^N(v));
        P=P\{v};
        X=X U {v};
    }
}
*/

\end{lstlisting}
\subsection{dinic 2}
\begin{lstlisting}
#include<vector>
#include<string.h>
#define FN 5010 
#define FM 500010
#define INF 1023456789
using namespace std;
struct E {
    int k,c;
    E(){}
    E( int _k, int _c ):k(_k),c(_c){}
} es[FM*2];

struct Flow {
    int n,m,dis[FN],ptr[FN];
    int qq[FN],qr,ql;
    vector<int> e[FN];
    void init( int _n ) {
        n=_n; m=0;
        for ( int i=0; i<n; i++ ) e[i]=vector<int>();
    }
    void add( int a, int b, int c ) {
        e[a].push_back(m); es[m]=E(b,c); m++;
        e[b].push_back(m); es[m]=E(a,0); m++;
    }
    bool BFS() {
        memset(dis,-1,n*sizeof(int));
        ql=qr=0;
        qq[qr++]=0;
        dis[0]=0;
        while ( ql!=qr && dis[n-1]==-1 ) {
            int p=qq[ql++];
            for(int i=0;i < (int)e[p].size(); i++) {
                E ee=es[ e[p][i] ];
                if ( ee.c==0 || dis[ee.k]!=-1 ) continue;
                dis[ee.k]=dis[p]+1;
                qq[qr++]=ee.k;
            }
        }
        return dis[n-1]!=-1;
    }
    int go( int p, int c ) {
        if ( p==n-1 ) return c;
        int tmp;
        for(int i=ptr[p]; i<(int)e[p].size(); i++) {
            E &ee=es[e[p][i]];
            if ( ee.c==0 || dis[p]+1!=dis[ee.k] ) continue;
            tmp=go(ee.k,min(c,ee.c));
            if(tmp != 0){
                ee.c-=tmp; es[e[p][i]^1].c+=tmp;
                ptr[p]=i;
                return tmp;
            }
        }
        ptr[p] = (int)e[p].size();
        return 0;
    }
    LL maxflow() {
        LL ret=0;
        while ( BFS() ){
            for(int i=0;i<n;i++)ptr[i]=0;
            while(true){
                int tmp=go(0,INF);
                if(tmp)ret+=tmp;
                else break;
            }
        }
        return ret;
    }
} flow;
int main(){
    return 0;
}

\end{lstlisting}
\subsection{dinic}
\begin{lstlisting}
const int INF = 100000;
int N, M, K;
int map[105][105];
int dist[105];
// 1 ~ N: origianl
// 0 : S, N + 1 T
bool BFS() {
	memset(dist, -1, sizeof(dist));
	dist[0] = 0;
	std::queue< int > Q;
	Q.push(0);
	while (!Q.empty()) {
		int now = Q.front();
		Q.pop();
		for (int i = 0;i <= N + 1; ++i)
			if (dist[i] == -1 && map[now][i]) {
				Q.push(i);
				dist[i] = dist[now] + 1;
			}
	}
	return dist[N + 1] != -1;
}
int DFS(int node, int flow) {
	if (node == N + 1) {
		return flow;
	}
	int ret = 0;
	for (int i = 0;i <= N + 1; ++i)
		if(map[node][i] && dist[i] == dist[node] + 1) {
			int bneck = min(flow, map[node][i]);
			int tmp = DFS(i, bneck);
			map[node][i] -= tmp;
			map[i][node] += tmp;
			flow -= tmp;
			ret += tmp;
		}
	return ret;
}
int maxFlow() {
	int totalFlow = 0;
	while (BFS()) {
		totalFlow += DFS(0, INF);
	}
	return totalFlow;
}

\end{lstlisting}
\subsection{DMST}
\begin{lstlisting}
const int NO_SOLUTION = -1;
class Edge { public:
    int v,u,l;
};
const int MAXEDGE=500000;
const int MAXNUM=600;
const int INF=1<<29;
Edge e[MAXEDGE],pred[MAXNUM];
int incycle[MAXNUM]={0},cycid;
bool contracted[MAXNUM];
int dmst(int s,int n,int m){
    int v,u,i,cost=0;
    for(i=0;i<n;i++) { pred[i].v=-1; contracted[i]=0; }
    while(1) {
        // find a uncontracted node with no in arc: v
        for(v=0;v<n;v++) if(v!=s&&!contracted[v]&&pred[v].v==-1) break;
        if(v==n) return cost; // done // find least-weighted in arc
        pred[v].l=INF;
        for(i=0;i<m;i++)
            if(e[i].u==v&&e[i].l<pred[v].l) pred[v]=e[i];
        if(pred[v].l==INF) return -1;
        // append arc, check cycle
        cost+=pred[v].l;
        for(u=pred[v].v;u!=v&&u!=-1;u=pred[u].v);
        if(u==-1) continue;
        // trace and contract nodes in cycle
        incycle[v]=++cycid;
        for(u=pred[v].v;u!=v;u=pred[u].v) {
            contracted[u]=1;
            incycle[u]=cycid;
        }
        // update arc costs into the cycle
        for(i=0;i<m;i++)
            if(incycle[e[i].v]!=cycid&&incycle[e[i].u]==cycid)
                e[i].l-=pred[e[i].u].l;
        // contract: update labels
        pred[v].v=-1;
        for(i=0;i<m;i++) {
            if(incycle[e[i].v]==cycid) e[i].v=v;
            if(incycle[e[i].u]==cycid) e[i].u=v;
            if(e[i].v==e[i].u) e[i--]=e[--m];
        }
        for(i=0;i<n;i++) {
            if(contracted[i]) continue;
            if(pred[i].v>=0&&incycle[pred[i].v]==cycid) pred[i].v=v;
        }
    }
}
int main(){}
\end{lstlisting}
\subsection{general matching}
\begin{lstlisting}
const int N=510;
VI G[N];
int fa[N],mh[N];
char color[N];
deque<int>path[N];
queue<int>Q;
int greedy_match(int n)  {
	int ans=0;
	FOR1(i,n)if(mh[i]==0)
		FOR(j,G[i].size())
			if(mh[G[i][j]]==0) {
				mh[G[i][j]]=i;
				mh[i]=G[i][j];
				ans++;
				break;
			}
	return ans;
}
void recordPath(int x,int y,int ba){
	for(int i=ba+1;i<path[x].size();i++){
		int p=path[x][i];
		if(color[p]==1){
			path[p]=path[y];
			path[p].insert(path[p].end(),path[x].rbegin(),path[x].rend()-i);
			color[p]=0;
			Q.push(p);
		}
	}
}
bool findPath(int st,int n){
	while(!Q.empty())Q.pop();
	FOR1(i,n){
		color[i]=-1;
		path[i].clear();
	}
	Q.push(st);
	color[st]=0;
	path[st].pb(st);
	while(!Q.empty()){
		int x=Q.front();Q.pop();
		FOR(i,G[x].size()){
			int y=G[x][i];
			if(mh[y]==y||y==st)continue;
			if(mh[y]==0){
				for(int i=1;i<path[x].size();i+=2){
					mh[path[x][i]]=path[x][i-1];
					mh[path[x][i-1]]=path[x][i];
				}
				mh[x]=y;mh[y]=x;
				return true;
			}
			if(color[y]==-1){
				color[y]=1;
				int &my=mh[y];
				color[my]=0;
				path[my]=path[x];
				path[my].pb(y);
				path[my].pb(my);
				Q.push(my);
			}
			else if(color[y]==0){
				int base=0;
				while(base<path[x].size()&&base<path[y].size()&&path[x][base]==path[y][base])
					base++;
				base--;
				recordPath(x,y,base);
				recordPath(y,x,base);
			}
		}
	}
	return false;
}
int general_match(int n){
	FOR1(i,n)mh[i]=0;
	int ans=greedy_match(n);
	FOR1(i,n)if(mh[i]==0){
		if(findPath(i,n))ans++;
		else mh[i]=i;
	}
	return ans;
}
int main(){
	int n,m;
	RI(n,m);
	FOR1(i,n)G[i].clear();
	while(m--){
		RID(x);RID(y);
		G[x].pb(y);
		G[y].pb(x);
	}
	printf("%d\n",general_match(n));
}


\end{lstlisting}
\subsection{kth-shortest path}
\begin{lstlisting}
#include <cstdio>
#include <queue>
const int N=1010,M=100010,MAX=1<<30;
struct edge{
	int p,c;
	edge *next;
}h[N],rh[N],v[M*2];
int vtop;
int dis[N];
void add(int x,int y,int c,edge h[]){
	edge *tmp=&v[vtop++];
	tmp->p=y;tmp->c=c;
	tmp->next=h[x].next;
	h[x].next=tmp;
}
struct dunit{
	int x,d;
	bool operator<(const dunit &A)const
	{return d>A.d;}
};
std::priority_queue<dunit>Q;
void Dij(int s,int n,edge h[]){
	for(int i=1;i<=n;i++)
		dis[i]=MAX;
	Q.push((dunit){s,0});
	dis[s]=0;
	bool flag[N]={};
	while(!Q.empty()){
		dunit a=Q.top();Q.pop();
		if(flag[a.x])continue;
		flag[a.x]=true;
		for(edge *tmp=h[a.x].next;tmp;tmp=tmp->next)
			if(dis[tmp->p] > a.d+tmp->c){
				dis[tmp->p]=a.d+tmp->c;
				Q.push((dunit){tmp->p,dis[tmp->p]});
			}
	}
}
struct unit{
	int x,d;
	bool operator<(const unit &A)const{
		return d+dis[x]>A.d+dis[A.x];
	}
};
std::priority_queue<unit>pQ;
int Astar(int s,int t,int n,int k){
	int vis[N]={};
	pQ.push((unit){s,0});
	while(!pQ.empty()){
		unit a=pQ.top();pQ.pop();
		if(vis[a.x]>k)continue;
		vis[a.x]++;
		if(vis[t]==k)return a.d;
		for(edge *tmp=h[a.x].next;tmp;tmp=tmp->next)
			pQ.push((unit){tmp->p,a.d+tmp->c});
	}
	return -1;
}
int main(){
	int n,m,x,y,c;
	while(~scanf("%d%d",&n,&m)){
		vtop=0;
		for(int i=1;i<=n;i++)
			h[i].next=rh[i].next=NULL;
		while(m--){
			scanf("%d%d%d",&x,&y,&c);
			add(x,y,c,h);
			add(y,x,c,rh);
		}
		int s,t,k;
		scanf("%d%d%d",&s,&t,&k);
		if(s==t)k++;
		Dij(t,n,rh);
		if(dis[s]==MAX)puts("-1");
		else printf("%d\n",Astar(s,t,n,k));
	}
}

\end{lstlisting}
\subsection{steiner tree}
\begin{lstlisting}
#include <cstdio>
#include <algorithm>
const int N=31,INF=1<<28;
int conn[N][N];
int DP[1<<8][N];
template<class T>inline void miz(T &a,T b){if(a>b)a=b;}
int steiner_tree(int T[],int top,int n){
	for(int k=0;k<n;k++)
		for(int i=0;i<n;i++)
			for(int j=0;j<n;j++)
				miz(conn[i][j],conn[i][k]+conn[k][j]);
	for(int i=0;i<(1<<top);i++)
		for(int j=0;j<n;j++)
			DP[i][j]=INF;
	for(int i=0;i<top;i++)
		for(int j=0;j<n;j++)
			DP[1<<i][j]=conn[T[i]][j];
	for(int i=2;i<(1<<top);i++)
		if(i&(i-1)){//__count(i)>1
			for(int j=0;j<n;j++)
				for(int sub=(i-1)&i;sub>0;sub=(sub-1)&i)
					miz(DP[i][j],DP[sub][j]+DP[i^sub][j]);
			for(int j=0;j<n;j++)
				for(int k=0;k<n;k++)
					miz(DP[i][j],DP[i][k]+conn[k][j]);
		}
	int ans=INF;
	for(int i=0;i<n;i++)
		miz(ans,DP[(1<<top)-1][i]);
	return ans;
}
int main(){
	int n,m;
	while(~scanf("%d%d",&n,&m)&&(n||m)){
		for(int i=0;i<n;i++)
			for(int j=i+1;j<n;j++)
				conn[i][j]=conn[j][i]=INF;
		while(m--){
			int cost,x,y;
			scanf("%d%d%d",&x,&y,&cost);
			miz(conn[x][y],cost);
			miz(conn[y][x],cost);
		}
		int k,T[N];
		scanf("%d",&k);
		for(int i=0;i<k;i++)
			scanf("%d",&T[i]);
		printf("%d\n",steiner_tree(T,k,n));
	}
}

\end{lstlisting}
\subsection{Stoer-Wagner Algorithm }
\begin{lstlisting}
//min-cut N^3
const int N=160;
int n,m;
struct node{int p,c;struct node* next;
	node(){}
	node(int P,int C,node *n):p(P),c(C),next(n){}
};
node h[N],v[N*N*N];
int conn[N][N],line[N],top,dis[N];
bool flag[N],use[N];
int vtop;
void add(int x,int y,int c){
	if(conn[x][y]){conn[x][y]+=c;return;}
	h[x].next=&(v[vtop++]=node(y,c,h[x].next));
	conn[x][y]=c;
}
void DFS(int x){
	flag[x]=true;
	for(node* tmp=h[x].next;tmp;tmp=tmp->next)
		if(!flag[tmp->p])
			DFS(tmp->p);
}
void MAS(int nn){
	mst(flag);mst(dis);
	priority_queue<PII>Q;
	for(int i=1;i<=n;i++)
		flag[i]=use[i];
	top=0;
	for(int i=1;i<=n;i++)
		if(!flag[i]){
			dis[i]=1;
			Q.push(mpr(dis[i],i));
			break;
		}
	while(nn){
		int x=Q.top().Y;Q.pop();
		if(flag[x])continue;
		nn--;
		line[top++]=x;
		flag[x]=1;
		for(node *tmp=h[x].next;tmp;tmp=tmp->next)
			if(!flag[tmp->p]){
				dis[tmp->p]+=conn[x][tmp->p];
				Q.push(mpr(dis[tmp->p],tmp->p));
			}
	}
}
int main(){
	int T,x,y,c;
	scanf("%d",&T);
	while(T--){
		scanf("%d%d",&n,&m);
		vtop=0;
		mst(conn);
		for(int i=1;i<=n;i++)
			h[i].next=NULL;
		while(m--){
			scanf("%d%d%d",&x,&y,&c);
			add(x,y,c);
			add(y,x,c);
		}
		mst(flag);
		DFS(1);
		bool connect=true;
		for(int i=1;i<=n;i++)
			if(!flag[i])connect=false;
		if(!connect){puts("0");continue;}
		int ans=2147483647,nn=n;
		mst(use);
		while(nn>1){
			MAS(nn);
			int x=line[top-1],y=line[top-2];
			miz(ans,dis[x]);
			use[x]=1;
			for(node *tmp=h[x].next;tmp;tmp=tmp->next)
				if(tmp->p!=y){
					add(y,tmp->p,conn[x][tmp->p]);
					add(tmp->p,y,conn[x][tmp->p]);
				}
			nn--;
		}
		printf("%d\n",ans);
	}
}

\end{lstlisting}
\section{math}
\subsection{extendedGCD}
\begin{lstlisting}
void exgcd(int a,int b,int &i,int &j,int &g){
	if(b==0){g=a;i=1;j=0;return;}
	exgcd(b,a%b,j,i,g);
	j-=(a/b)*i;
}//a*p+b*q=g, min(|p|+|q|)
int main(){
	RI(a,b);
	int p,q,g;
	exgcd(a,b,p,q,g);
	printf("%d %d %d\n",p,q,g);
}

\end{lstlisting}
\subsection{FFT}
\begin{lstlisting}
#include <bits/stdc++.h>
using namespace std;
const int S=5,N=1<<20;
typedef long double LD;
const LD pi=acos(-1.0);
struct C{
	LD a,b;
	C(){}
	C(LD _a){a=_a;b=0;}
	C(LD _a,LD _b){a=_a;b=_b;}
	LD& real(){return a;}
	inline C operator*(const C &A)const{
		return C(a*A.a-b*A.b,a*A.b+b*A.a);
	}
	inline C operator-(const C &A)const{
		return C(a-A.a,b-A.b);
	}
	inline void operator+=(const C &A){
		a+=A.a;b+=A.b;
	}
	inline C operator/(const double &N)const{
		return C(a/N,b/N);
	}
};
char s[N+10],c[N+10];
C x[N+10],y[N+10],X[N+10],Y[N+10];
long long ans[N];
void change(char s[],C x[]){
	int n=strlen(s),top=0;
	for(int i=0;i<n;i++){
		for(int j=min(i+S-1,n-1);j>=i;j--)
			x[top].real()=x[top].real()*10+s[n-j-1]-'0';
		top++;
		i+=S-1;
	}
}
void FFT(C x[],int n,int s,C X[],const int &k){
	if(n==1){
		X[0]=x[0];
		return ;
	}
	n>>=1;
	FFT(x,n,s<<1,X,k);
	FFT(x+s,n,s<<1,X+n,k);
	double w=-pi*k/n;
	for(int i=n-1;i>=0;i--){
		C off=X[i+n]*C(cos(w*i),sin(w*i));
		X[i+n]=X[i]-off;
		X[i]+=off;
	}
}
int main(){// calc (Bigint)s[]*c[], LD->D or LL->int to avoid TLE
	int T;scanf("%d",&T);
	while(T--){
		scanf("%s%s",s,c);
		int N=1<<(32-__builtin_clz((strlen(s)+strlen(c))/S+1));
		if(N>(1<<20))
			N=1<<20;
		for(int i=0;i<N;i++)
			x[i]=y[i]=0;
		change(s,x);
		change(c,y);
		FFT(x,N,1,X,1);
		FFT(y,N,1,Y,1);
		for(int i=0;i<N;i++)
			X[i]=(X[i]*Y[i])/(double)N;
		FFT(X,N,1,x,-1);
		for(int i=0;i<N;i++)
			ans[i]=x[i].real()+1e-1;
		for(int i=0;i<N;i++){
			ans[i+1]+=ans[i]/100000;
			ans[i]%=100000;
		}
		int last=N-1;
		while(last>=1&&ans[last]==0)last--;
		printf("%lld",ans[last]);
		for(int i=last-1;i>=0;i--)
			printf("%05lld",ans[i]);
		puts("");
	}
}

\end{lstlisting}
\subsection{inverse}
\begin{lstlisting}
int inv(LL x, LL m) {return x==1?1:(1-m*inv(m%x,x))/x%m;}//4 7 return -2
\end{lstlisting}
\subsection{mod-log}
\begin{lstlisting}
/* a^x = b(mod m), need (a,m)=1 */
int logM(int a,int b,int m) {
	map<int,int>x;//a^0,...,a^(n-1)
	int n=sqrt(m);
	int t=1;
	for(int i=0;i<n;i++) {
		if(t==b)return i;
		if(!x.count(t))x[t]=i;
		t=mul(t,a,m);//t=t*a%m
	}
	//a^n..a^(2n-1)
	//a^2n..a^(3n-1)
	//...
	int f=inv(t,m);
	for(int i=0;i<n+1;i++) {
		if(x.count(b))return i*n+x[b];
		b=mul(b,f,m);
	}
	return -1;
}

\end{lstlisting}
\subsection{xorOfArithSeq}
\begin{lstlisting}
//xor of n ArithSeq
#include <cstdio>
const int N=200100;
LL A[N],B[N],C[N];
int bit(int a,int s){
	return (a>>s) &1;
}
int K(LL a,LL b,LL c){
	return (b-a)/c;
}
LL fun(LL x,LL a,LL b,LL n){
	LL res=0,k=(a/b)&1;
	if(n<0)return 0;
	a%=b;
	if(n&1)
		res=((n+1>>1)&1);
	else
		res=((n/2)&1);
	res=(res*k)&1;
	if(a==0){
		res+=((x/b)&1)*((n+1)&1);
        return res&1;
	}
	res+=((x/b)&1)+(n&1)*(((x+n*a)/b)&1)+fun(b-x%b-1,b,a,(x+n*a)/b-x/b-1);
	return res&1;
}
int main(){
	int n;
	while(~scanf("%d",&n)){
		for(int i=0;i<n;i++){
			int a,b,c;
			scanf("%d%d%d",&a,&b,&c);
			A[i]=a;//first
			B[i]=b;//last
			C[i]=c;//diff
		}
		unsigned ans=0;
		for(int i=0;i<n;i++){
			unsigned x=0;
			LL add=0;
			for(int j=0;j<32;j++)
				x^=fun(A[i],C[i],1LL<<j,K(A[i],B[i],C[i]))<<j;
			ans^=x;
		}
		printf("%u\n",ans);
	}
}

\end{lstlisting}
\section{string}
\subsection{Aho-Corasick Algorithm}
\begin{lstlisting}
#include <bits/stdc++.h>
using namespace std;
const int N=160;
struct node{
	node *next[27],*fail;
	int idx,cnt;
	int vidx;
	node(int vtop=-1,int _idx=-1){
		for(int i=0;i<27;i++)next[i]=NULL;
		fail=NULL;
		idx=_idx;cnt=0;
		vidx=vtop;
	}
}*root,v[N*70];
char s[N][80],tar[1000010];
int stov[N];
int vtop;
node* newnode(){
	v[vtop]=node(vtop);vtop++;
	return &v[vtop-1];
}
void insert(char s[],int idx){
	node *now=root;
	int n=strlen(s);
	for(int i=0;i<n;i++){
		if(now->next[s[i]-'a']==NULL)
			now->next[s[i]-'a']=newnode();
		now=now->next[s[i]-'a'];
	}
	stov[idx]=now->vidx;
	if(now->idx==-1)
		now->idx=idx;
	else
		stov[idx]=now->vidx;
}
queue<node*>Q;
void make_fail(){
	while(!Q.empty())Q.pop();
	root ->fail=root;
	Q.push(root);
	while(!Q.empty()){
		node *now=Q.front();Q.pop();
		for(int i=0;i<26;i++)
			if(now->next[i]!=NULL){
				node *tmp=now;
				bool ok=false;
				while(tmp!=root){
					tmp=tmp->fail;
					if(tmp->next[i]!=NULL){
						now->next[i]->fail=tmp->next[i];
						ok=true;
						break;
					}
				}
				if(!ok)now->next[i]->fail=root;
				Q.push(now->next[i]);
			}
	}
}
void update(node *now){
	if(now==root)return;
	now->cnt++;
	update(now->fail);
}
void string_match(char s[]){
	int n=strlen(s);
	s[n++]='z'+1;
	node *now=root;
	for(int i=0;i<n;i++){
		while(now!=root&&now->next[s[i] -'a']==NULL)
			now=now->fail;
		if(now->next[s[i]-'a']!=NULL)
			now=now->next[s[i]-'a'];
		update(now);
	}
}
int main(){
	int n;
	while(~scanf("%d",&n)&&n){
		vtop=0;
		root=newnode();
		for(int i=0;i<n;i++){
			scanf("%s",s[i]);
			insert(s[i],i);
		}
		make_fail();
		scanf("%s",tar);
		string_match(tar);
		int Max=0;
		for(int i=0;i<n;i++)
			if(Max<v[stov[i]].cnt)
				Max=v[stov[i]].cnt;
		printf("%d\n",Max);
		for(int i=0;i<n;i++)
			if(Max==v[stov[i]].cnt)
				printf("%s\n",s[i]);
	}
}

\end{lstlisting}
\subsection{KMP}
\begin{lstlisting}
int fail[N];
void KMP(char t[],char p[]) {
	int n=strlen(t),m=strlen(p);
	if(m>n)return;
	for(int i=1,j=fail[0]=-1;i<m;i++){
		while(j>=0 && p[j+1]!=p[i])
			j=fail[j];
		if(p[j+1]==p[i])j++;
		fail[i]=j;
	}
	for(int i=0,j=-1;i<n;i++) {
		while(j>=0 && p[j+1]!=t[i])
			j=fail[j];
		if(p[j+1]==t[i])j++;
		if(j==m-1) {
			printf("p occur at %d\n",i-m+1);
			j=fail[j];
		}
	}
}

\end{lstlisting}
\subsection{longest palindrome}
\begin{lstlisting}
const int N=500001;
char s[N*2],c[N];
int pal[N*2],R;
int match(int a,int b,int m){
	int i=0;
	while(a-i>=0&&b+i<m&&s[a-i]==s[b+i])i++;
	return i;
}
int longest_palindrome(char c[]){
	int p,ii,m=strlen(c);
	for(int i=0;i<m;i++){
		s[i*2]='@';
		s[i*2+1]=c[i];
	}
	s[m*2]='@';m=m*2+1;
	pal[0]=0;p=R=0;
	//p=max_idx{pal[i]+i},R=max{pal[i]+i}
	for(int i=1;i<m;i++){
		ii=p-(i-p); //reflextion point
		int n=R-i;  //distance of i and pal[p]+p 
		if(i>R){ //overflow R
			pal[i]=match(i-1,i+1,m);
			R=i+pal[p=i];
		}
		else if(pal[ii]==n){ //exactly at n
			pal[i]=n+match(i-n-1,i+n+1,m);
			R=i+pal[p=i];
		}
		else pal[i]=min(pal[ii],n);
	}
	int ans=0;
	for(int i=0;i<m;i++)
		maz(ans,pal[i]);
	return ans;
}
int main(){
	scanf("%s",c);
	printf("%d\n",longest_palindrome(c));
}

\end{lstlisting}
\subsection{SAIS}
\begin{lstlisting}
struct SAIS{
	static const int N=100100;
	//S[n-1] must be the smallest character
	//MAX alphabet must < N
	int S[N*2],SA[N*2],hei[N];// hei[i]=maxlen of SA[i],SA[i-1]
	bool _iss[N*2];
	int _p[N*2],_pb[N*2],cnt[N],qe[N];
	void build(int/*char*/ s[],int n){
		for(int i=0;i<n;i++)S[i]=s[i];
		suffixArray(n);
		//mkhei(n);
	}
	inline int operator[](int i){return SA[i];}
	void isort(int n,int *s,int *sa,bool iss[],int p[],int pc){
		int a=0,i;
		for(int i=0;i<n;i++)a=max(a,s[i]);a++;
		memset(cnt,0,sizeof(int)*a);
		for(int i=0;i<n;i++)cnt[s[i]]++;
		qe[0]=cnt[0];for(int i=1;i<a;i++)qe[i]=qe[i-1]+cnt[i];
		memset(sa,-1,sizeof(int)*n);
		for(int i=pc-1;i>=0;i--)sa[--qe[s[p[i]]]]=p[i];
		qe[0]=0;for(int i=1;i<a;i++)qe[i]=qe[i-1]+cnt[i-1];
		for(int i=0;i<n;i++)if(sa[i]>0 && !iss[sa[i]-1])sa[qe[s[sa[i]-1]]++]=sa[i]-1;
		qe[0]=cnt[0];for(int i=1;i<a;i++)qe[i]=qe[i-1]+cnt[i];
		for(int i=n-1;i>=0;i--)if(sa[i]>0&&iss[sa[i]-1])sa[--qe[s[sa[i]-1]]]=sa[i]-1;
	}
	bool eq(int *s,bool iss[],int pp[],int pb[],int pc,int x,int p){
		if(pb[p]==pc-1 || pb[x]==pc-1 || pp[pb[p]+1]-p!=pp[pb[x]+1]-x)return 0;
		for(int j=0;j<=pp[pb[p]+1]-p;j++)if(s[j+p]!=s[j+x]||iss[j+p]!=iss[j+x])return 0;
		return 1;
	}
	void suffixArray(int n,int a1=0){
		int *s=S+a1,*sa=SA+a1,*pp=_p+a1,*pb=_pb+a1,pc=0;
		bool *iss=_iss+a1;
		iss[n-1]=1;
		for(int i=n-2;i>=0;i--)iss[i]=s[i]<s[i+1]||(s[i]==s[i+1]&&iss[i+1]);
		for(int i=1;i<n;i++)if(iss[i]&&!iss[i-1])pp[pb[i]=pc++]=i;
		isort(n,s,sa,iss,pp,pc);
		int p=-1,c=-1;
		for(int i=0;i<n;i++){
			int x=sa[i];
			if(x&&iss[x]&&!iss[x-1]){
				if(p==-1||!eq(s,iss,pp,pb,pc,x,p))c++;
				s[n+pb[p=x]]=c;
			}
		}
		if(c==pc-1)for(int i=0;i<pc;i++)sa[n+s[n+i]]=i;
		else suffixArray(pc,a1+n);
		for(int i=0;i<pc;i++)pb[i]=pp[sa[n+i]];
		isort(n,s,sa,iss,pb,pc);
	}
	void mkhei(int n){
        static int r[N];
        for(int i=0;i<n;i++) r[SA[i]] = i;
        hei[0] = 0;
        for(int i=0;i<n;i++)if(r[i]) {
            hei[r[i]] = i>0 ? max(hei[r[i-1]] - 1, 0) : 0;
            while(S[i+hei[r[i]]] == S[SA[r[i]-1]+hei[r[i]]]) hei[r[i]]++;
        }
    }
}sa;
int main(){
	while(gets(s)){
		int n=strlen(s);
		sa.build(s,n+1);
		FOR1(i,n)
			printf("%d\n",sa[i]);
	}
}

\end{lstlisting}
\subsection{suffix array}
\begin{lstlisting}
#include <bits/stdc++.h>
const int MAX=200001;
char s[MAX];
int SA[MAX],A[MAX],B[MAX],*r1=A,*r2=B,h,len;
void swap(){int *tmp=r1;r1=r2;r2=tmp;}
bool cmp(int i,int j){return
(r1[i]<r1[j])||(r1[i]==r1[j]&&r1[i+h]<r1[j+h]);}
int main(){
	gets(s);
	len=strlen(s)+1;
	for(int i=0;i<len;i++)SA[i]=i,r1[i]=s[i];
	for(;h<=len;){
		std::sort(SA,SA+len,cmp);
		for(int i=0,j=0;i<len;i++){
			if(i&&cmp(SA[i-1],SA[i]))j++;
			r2[SA[i]]=j;
		}
		swap();
		h+=h?h:1;
	}
	for(int i=1;i<len;i++,puts(""))
		for(int j=SA[i];j<len;j++)
			printf("%c",s[j]);
}

\end{lstlisting}
\subsection{SuffixAutomation}
\begin{lstlisting}
// SuffixAutomation
// each node is effectively an end-set, there are at most 2L-1 nodes.
// the end-set corresponding to a nodes parent is always its supet-set
// primary edges form tree, (reversed) par link form a (different) tree.
class State {
public:
    int len;
    State *par,*go[26];
    State(int _len=0):len(_len) {
        memset(go,0,sizeof(go));
    }
    ~State() {
        for(int i=0;i<26;i++)
            if(go[i]&&go[i]->len==len+1)
                delete go[i];
    }
};

class SuffixAutomation {
public:
    State root,*last;
    SuffixAutomation() {init();}
    void init() {
        root.par=NULL; root.len=0;last=&root;
    }
    void extend(int w) {
        State *cp=last;
        State *np=new State(cp->len+1);
        while(cp && cp->go[w]==NULL) {
            cp->go[w]=np;
            cp=cp->par;
        }
        if(!cp) np->par=&root;
        else {
            State *cq=cp->go[w];
            if(cq->len==cp->len+1) {
                np->par=cq;
            } else {
                State *nq=new State(cp->len+1);
                memcpy(nq->go,cq->go,sizeof(cq->go));
                nq->par=cq->par; cq->par=nq; np->par=nq;
                while(cp && cp->go[w]==cq) {
                        cp->go[w]=nq;
                        cp=cp->par;
                }
            }
        }
        last=np;
    }
    void extend(char *str) {
        for(int i=0;str[i];i++)
            extend(str[i]-'a');
    }
};
\end{lstlisting}
\end{document}
